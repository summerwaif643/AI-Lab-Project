\documentclass{article}
\usepackage{listings}
\usepackage{amssymb}
\usepackage{amsfonts} 
\usepackage{amsmath}
\usepackage{multicol}
\usepackage{graphicx}
\usepackage{url}
\usepackage{latexsym}
\usepackage{verbatim}
\usepackage{hyperref}

\newtheorem{definition}{Definition}

\title{AI Lab project: Colorizing images using PyTorch and common AIML techniques  }
\author{Alessandro Candelori, Davide di Trocchio}


\begin{document}
\maketitle
\newpage
\tableofcontents
\newpage


\section{Introduction}
We aim to develop a model in pytorch for image coloring. Its main goal is to colour a never seen 
before, black and white image, ranging from old photographs, decolourized images, sketches and 
black and white media. 
\\ 
The task at hand is a regression model built with inspiration from models like the 
\href{https://arxiv.org/pdf/1611.07004.pdf}{Image-to-Image Transaltion}
and 
\href{https://arxiv.org/pdf/1603.08511.pdf}{Colorful Image Colorization}. 
In particular, we took inspiration from the latter and got really similiar results to their regression
models. 
As we continued to study further, we managed to find many other methods of coloring images, but 
we choose the simplest one, as the other needed too much control and very large datasets, that 
we could not include due to the nature of this project. 
\\
We stressed a classical software development approach as we wanted to study how we could deploy 
AI apps, as opposed to publish our research to a simple Google Colab page or a Jupyter Notebook. 
We like to think that we succeeded in creating a cool, user-level application that could be 
understood and used by everyone. 

\section{Method}
As said in the introduction, we employ a ResNet regression structure to learn related information
from a converted grayscale image. Each image is transposed to a LAB format before learning, extracting 
the features from the grayscale image (using the L channel).

\subsection{LAB format and image processing}
Before processing any image, we aim to bring them to a given resolution ($256 \times 256$) and then 
convert them from the RGB format to the LAB one. This is done because the RGB format makes it more 
difficult to differentiate colors from the grayscale, whereas in the LAB format we have 
immediately, without computation, the Lightness channel and the color channels (AB) divided from the 
start. 
\pagebreak

\begin{figure}[htp]

    %TODO: insert here all subsequents 
    \centering
    \includegraphics[width=.7\textwidth]{tex_images/lab.jpg}\hfill

    
    \caption{The LAB color space. L for including lightness and 
    AB for every other space}
    \label{fig:figure 1.}
    
    \end{figure}

We will apply regression on the L channel and use the AB channels as the 
ground truth. 

\subsection{The net}
We make use of the ResNet-18, which is an image classification network with 18
layers, of which we will only use 6 modifying the first one to accept grayscale
images. The other layers are simple Convolutional, Batchnorm and upsample 
layers in the order presented inside the code. 

\begin{figure}[htp]

    %TODO: insert here all subsequents 
    \centering
    \includegraphics[width=.8\textwidth]{tex_images/net.png}\hfill

    
    \caption{ResNet Model including image processing steps}
    \label{fig:figure 1.}
    
    \end{figure}


\subsection{Loss function}
As the loss function, we work with the simple Mean Squared error. We try to 
minimize the squared distance between the color value we try to predict and the 
actual ground truth color value. 
$$ MSE = \frac{1}{n} \sum^{n}_{i=1}(Y_i - \hat{Y_i})^2$$ 

\subsection{Activation Function}
Since we're using ResNet18, our model is bound to use ReLu. We tried using 
newer techniques such as Mish, but after observing few results we can safely 
say that Mish performs too drastically on this model. 
% Insert image 1, 2 here
\begin{figure}[htp]

    %TODO: insert here all subsequents 
    \centering
    \includegraphics[width=.5\textwidth]{tex_images/mish.jpeg}\hfill
    \includegraphics[width=.5\textwidth]{tex_images/ReLU.jpeg}\hfill
    
    \caption{On the left the model using mish, on the right ReLU}
    \label{fig:figure 1.}
    
    \end{figure}

This is because the Mish activates earlier than the ReLU, leaving us with more rash
decisions. 

\subsection{Optimizer}
We use the Adam optimizer simply because it is one of the most officient optimizer 
available to us.  

\subsection{The dataset}
The dataset we used is made of 40,000 colored images in $256 \times 256$ format.
Nearly every dataset made of images in whatever format works fine, as we then apply
our transformations on it and derive the grayscale image. 
The ones we used are a combination of MIT Places, various kaggle datasets ranging 
from people to cars to forests and such. As said before, nearly every dataset provides
useful in this situation. 


\subsection{Basic Code documentation}
The scope of the project was to create an AI app through standard 
software development techniques. Inside of the project you'll find
different files that are here described. 

\subsubsection{net.py}
$\text{net.py}$ Holds the core to train the entire application. It is made of two 
big functions, train and validate, which are then used when the model is created to 
train it. Each time we finish training an epoch, we validate it and store our loss. 
If the loss results lower than some other model, we create a "checkpoint" inside the 
same folder, in order to start working with that model in our $\text{frontend.py}$ 
Running this python file will start training with the settings found 
in it. 

\subsubsection{grayscale.py}
$\text{grayscale.py}$ is a utility class which supers torchvision dataset imageloader. 
This is done because each image before being loaded needs to be split into L and AB, 
providing the ground truth and the black and white images we use to train our model. 


\subsubsection{frontend.py}
$\text{frontend.py}$ holds the core of the application. Using the $\text{streamlit}$
python module, we create a simple web app which lets the end user upload a file and 
have it colored through a saved model. This is all done in real time speed and it also 
holds a simple cache which deletes the validated images once uploaded. It is runnable 
with: \\ 
\begin{verbatim}
    streamlit run frontend.py
\end{verbatim}
This brings the whole application to life as it is deployable to the internet to show a 
simple demo. 
Important! This file contains absolute paths and creates problems when working with 
relatives, as such, if you wish to run it, please consider updating all paths to 
include your preferred method. 

\subsubsection{colorization.py}
Colorization.py holds the model used. It is a simple class which holds the 6 ResNet18
layers. 

\subsubsection{averagemeter.py}
In averagemeter we have the $\text{AverageMeter}$ class. 
This is a utility class often found in Ai projects for taking 
averages. In the program it is used to update the batch time 
and the loss time when training and validating the model.

\subsubsection{dataset.py}
$\text{dataset.py}$ holds an utility class aimed towards helping organize an unorganized
image dataset, which is often the case when working with datasets deriving from 
kaggle. Running it will simply create two folders following a 70/30 splitting scheme.

\begin{verbatim}
\end{verbatim}

\section{Results and ending}
Given the nature of training the model on our own devices, our results were 
capped at about 25 epochs. It takes us, on average, about 10 hours to get this far,
but we plan nonetheless to bring a presentation to the exam training up to the maximum.
\\
What follows are some cherry picked results of some images we thought our model best 
colored. 

        \begin{figure}[htp]

            %TODO: insert here all subsequents 
            \centering
            \includegraphics[width=.5\textwidth]{tex_images/1.jpeg}\hfill
            \includegraphics[width=.5\textwidth]{tex_images/2.jpeg}\hfill
            
            \caption{On the left the ground truth, on the right the derived image}
            \label{fig:figure 1.}
            
            \end{figure}

\pagebreak

            \begin{figure}[htp]

                %TODO: insert here all subsequents 
                \includegraphics[width=.5\textwidth]{tex_images/6.jpeg}\hfill
                \includegraphics[width=.5\textwidth]{tex_images/5.jpeg}\hfill
                
                \caption{On the left the ground truth, on the right the derived image}
                \label{fig:figure 1.}
                
                \end{figure}

                \begin{figure}[htp]

                    %TODO: insert here all subsequents 
                    \includegraphics[width=.5\textwidth]{tex_images/7.jpeg}\hfill
                    \includegraphics[width=.5\textwidth]{tex_images/8.jpeg}\hfill
                    
                    \caption{On the left the ground truth, on the right the derived image}
                    \label{fig:figure 1.}
                    
                    \end{figure}
            As we can see, the model does a pretty good job considering it trained so 
            little. It has some problems determining where to put blues, which 
            could be seen as an overfit, but we produced some good results nonetheless.

            During the final presentation we will present a more trained model on 
            even more cases, such as photographs, sketches and black and white media.
            
            \pagebreak

            \section{Credits and links}
            \url{https://colab.research.google.com/github/moein-shariatnia/Deep-Learning/blob/main/Image%20Colorization%20Tutorial/Image%20Colorization%20with%20U-Net%20and%20GAN%20Tutorial.ipynb#scrollTo=mGqbGWUQhzmk}
            \url{https://haoyiq.com/project/eecs442/} \\
            \url{https://arxiv.org/pdf/1611.07004.pdf} \\
            \url{https://arxiv.org/pdf/1603.08511.pdf} \\
\end{document}
