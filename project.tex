\documentclass{article}
\usepackage{listings}
\usepackage{amssymb}
\usepackage{amsfonts} 
\usepackage{amsmath}
\usepackage{multicol}
\usepackage{graphicx}
\usepackage{url}
\usepackage{latexsym}
\usepackage{verbatim}
\usepackage{hyperref}

\newtheorem{definition}{Definition}

\title{AI Lab project: Colorizing images using PyTorch and common AIML techniques  }
\author{Alessandro Candelori, Davide di Trocchio}


\begin{document}
\maketitle
\newpage
\tableofcontents
\newpage


\section{Introduction}
We aim to develop a model in pytorch for black and white image coloring. Since we're dealing mostly 
with greyscale data, we aim to see how this model performs first on already encountered images, 
providing a metric for judgment based on on a ground truth, the black and white image and the model result. 
Then, we check its performance, and maybe work a little longer to see how it behaves on sketches, 
vectorized images and mangas. 

\section{Method}
Images usually are worked on by using the normal RGB standard to separate the three color values in 
their respective channels. Since we're dealing with black and white images, we're better off using the 
LAB standard, which stands for \begin{itemize}
    \item L = Lightness channel. This channel indicates how Lightness is affected inside 
            of the image. 
    \item A = 

\end{itemize}

we only hold the L value and try to predict the AB values 
based on what we've seen so far. At first glance, this may seem an expectation maximization problem
based on values AB.  

\subsection{The dataset}
The dataset we aimed to use was a simple image dataset acquired from Keggle. This keggle dataset 
is fairly small, containing over 5Gb of 200x200 images. Going any further with these number 
would have brought some problems in sharing our project, so we decided to go like this. 
Each and every image was then processed to obtain a suitable black and white image to train our 
CNN on.  

\section{Results and ending}


\section{Credits and links}
\url{https://lukemelas.github.io/image-colorization.html} \\ 
\url{https://colab.research.google.com/github/moein-shariatnia/Deep-Learning/blob/main/Image%20Colorization%20Tutorial/Image%20Colorization%20with%20U-Net%20and%20GAN%20Tutorial.ipynb#scrollTo=mGqbGWUQhzmk}



\end{document}
